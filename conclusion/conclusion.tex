\chapter*{Conclusion}
\addcontentsline{toc}{chapter}{Conclusion}

\setlength{\epigraphwidth}{0.6\textwidth}
\epigraph{When I choose to see the good side of things, I'm not being naive. It is strategic and necessary. It's how I've learned to survive through everything.}{\textsc{Waymond Wang}\\\textit{Everything Everywhere All at Once}}
\vspace{1.5cm}

Dans cette thèse, nous avons exploré la représentation du trill en utilisant différentes méthodes et différents jeux de données. La réalisation du \textit{r} est variable et dépend du niveau d'abstraction utilisé pour représenter le son. Les données proches du signal acoustique mettent en évidence plus de variation que les inventaires phonémiques. Tous les sons sont variables, mais les rhotiques le sont particulièrement \parencite{scobbieVariable2006}.
Nous avons notamment pris en compte des données issues des \textit{Illustrations of the IPA}, de Sound Comparisons et de nombreuses grammaires. Les différentes données permettent toutes de dire que le trill varie et mettent en avant la sous-représentation du tap et du flap.

\subsection*{La représentation du \textit{r} dans les transcriptions}

Dans les chapitres \ref{chap:jipa} et \ref{chap:soundcompa}, nous nous sommes intéressé à la représentation du trill dans les transcriptions. Ces transcriptions ne sont pas les nôtres et ont été faites par de nombreuses personnes avec un bagage plus ou moins important en linguistique de terrain et plus ou moins important en dialectologie.
Nos différents résultats montrent que plus la transcription est étroite, moins il y a de trills.
En revanche, les reconstructions de langues historiques ou de proto-langues sont associées avec plus de trills. Dans les langues considérées comme disposant d'un trill phonémique, c'est le symbole pour le tap/flap qui se retrouve plus fréquemment. Les transcriptions issues des \textit{Journal of the IPA} nous permettent de dire qu'il faut faire attention à ce que le segment \textit{r} englobe et qu'il faut prendre en compte l'influence des systèmes de transcriptions. L'intégration d'informations sur les transcripteurs qui ont collaboré au projet Sound Comparisons nous permet de contextualiser les transcriptions. L'étude de Sound Comparisons nous permet aussi de voir que les réalisations englobées par le symbole \textit{r} dans les langues d'Europe partagent des similarités avec celles du mapudungun, bien que les fréquences des allophones soient différentes.\\

\subsection*{La représentation du \textit{r} en acoustique}

Dans le \autoref{chap:acoustics}, nous avons exploré le trill à travers des enregistrements sonores. Nous avons segmenté et annoté des données sous Praat. Nous avons cherché des motifs dans les représentations acoustiques issues du signal acoustique. Deux méthodologies ont été utilisées. La première se basait sur des catégories larges et sur un grand échantillon de langues, la deuxième se basait sur un échantillon contenant quatre fois moins de langues, mais plus précise dans sa méthode pour l'obtention d'éléments composant le signal acoustique. Dans les deux méthodes, nous  avons dû procéder à une catégorisation manuelle. Bien que nous obtenons des différences, nous observons que les segments avec au moins deux contacts ne sont pas les allophones majoritaires. En dépit de travailler avec peu de données et avec des histoires racontées, nous avons montré que, pour les différentes langues de notre échantillon réduit, les motifs avec deux occlusions sont fréquemment en position intervocalique.

Travailler avec des données acoustiques nous a permis de montrer -- s'il en était besoin -- que les trills varient.
Même si chaque langue est représentée seulement par un.e locuteur/trice, cette variation est systématiquement présente.
Le segment à deux occlusions ne se retrouve pas uniquement avec les segments qui ont été labellisés comme trill, mettant en évidence toute la difficulté de la recherche du trill.

\subsection*{La représentation du \textit{r} dans les grammaires et les bases de données}

Le \autoref{chap:roughness} s'interroge sur l'interprétation des grammaires pour en obtenir des informations pour l'étude d'associations avec le trill d'un point de vue cross-linguistique. Nous avons pris l'exemple de l'article de \textcite{winterTrilledAssociatedRoughness2022} publié en Open Access, pour montrer qu'un recodage des données de l'étude d'origine mène à une conclusion différente. Nous avons parcouru et extrait de l'information sur les trills et les autres rhotiques dans plus de 600 ouvrages qui décrivent des langues du monde. Dire qu'une langue possède un /r/ trillé n'est pas simple et nécessite de prendre en compte de nombreux facteurs qui généralement ne sont pas explicites dans les grammaires. Nous avons montré que le concept de \textg{roughness} était vraisemblablement associé à la classe des rhotiques plutôt que spécifiquement au trill, et ce, parce que les grammaires ne sont pas conçues pour comparer les rhotiques qui y sont incluses. Le \textit{r} dans une langue pourra être décrit comme un trill, comme un tap, ou comme un flap en fonction de la personne qui décrit la langue. Le \autoref{chap:metagram} s'est intéressé aux indications données aux personnes étudiant et rapportant les sons des langues. Ce chapitre suggère que la rédaction d'une grammaire n'est pas une tâche évidente, et que cela s'accompagne de nombreux problèmes pour étudier les informations rapportées. Nous avons proposé une visualisation en graphe du trill et des autres rhotiques en fonction des concepts auxquels ils étaient associés. L'étude de processus phonologiques à travers l'analyse de base de données phonémiques nous a permis de dire que le symbole pour le trill est souvent présent même si sa réalisation n'est pas possible. Ainsi, il faut contextualiser les sources primaires.\\

Cherchant les trills, nous avons proposé de regarder la distribution des langues possédant un contraste entre deux rhotiques de différente longueur. Dans certains cas, on observe que l'une de ces deux rhotiques a effectivement l'allophone d'un des deux phonèmes qui est réalisé comme un trill. Cependant, ce n'est pas toujours le cas, et, le plus souvent, les sources primaires manquent pour accorder du crédit à cette hypothèse. De nombreuses grammaires ne sont pas trouvables rendant impossibles d'éventuelles réplications.\\

Les différentes représentations du \textit{r}, malgré tout le bruit existant, montrent tout de même que bien que le trill soit moins fréquent que l'on le pense, il reste présent dans les langues, souvent comme un allophone minoritaire.

\subsection*{Discussion}

La question initiale durant ce doctorat fut de comprendre pourquoi certaines personnes ne trillaient pas leurs \textit{r}. À partir des données que nous récupérions pour étudier la distribution du trill, de nouvelles questions émergeaient, comme celle de savoir pourquoi le \textit{r} trillé était tant décrit dans les langues du monde alors que sa maîtrise par les locuteurs/trices n'était pas aisée.
Pour répondre à cette question, il était nécessaire d'explorer la représentation des rhotiques, qui est biaisée en faveur du symbole \textit{r}, et de l'interprétation trillée du segment. Les résultats des différentes études que nous avons proposées suggéraient tous que les trills ne sont pas si fréquents en comparaison à d'autres variants comme le tap et le flap. Cependant, une faible fréquence ne voulait pas dire que le trill n'était pas utilisé par les utilisateurs d'une langue.\\

Dans certains cas, le trill est perçu comme un outil dont les locuteurs/trices peuvent se servir pour construire un sens autre que le sens linguistique. Cette thèse n'est pas une thèse de sociolinguistique, ainsi nous ne développerons pas en profondeur ce point. Chaque début de chapitre était accompagné, en épigraphe, d'un lien vers une chanson où on retrouve un trill (présent en \autoref{chap:ann_qrcode} pour la version papier).
Ainsi, nous avons pu trouver des trills même dans des langues où nous n'en attendions pas forcément.
La chanson peut favoriser le trill là où, dans la langue de tous les jours, il ne serait pas si commun (comme ici avec le japonais, le coréen, un dialecte du centre de la Chine, le thaï, le russe et le télougou). L'utilisation du trill dans certaines langues n'est pas dénuée de sens, on retrouve une certaine motivation à garder l'allophone [r] comme potentielle réalisation de la rhotique que ce soit à travers les chansons ou les onomatopées. Il s'agit d'un segment \textg{simil-\textit{r}} que nous ne serions pas surpris de voir apparaître dans des contextes marqués et dans des langues où il ne devrait pas être. \\
%Un sens social. 

%Dans certains cas, le trill est perçu comme un outil dont les locuteurs/trices peuvent se servir pour construire un sens autre que le sens linguistique. Cette thèse n'est pas une thèse de sociolinguistique, ainsi nous ne développerons pas en profondeur ce point. Chaque fin de chapitre était accompagnée, sur la dernière page, d'un lien vers une chanson où on retrouve un trill. Ainsi, dans des langues où nous n'attendions pas forcément de trill, nous en retrouvions. La chanson peut favoriser le trill là où dans la langue de tous les jours il ne serait pas si commun (comme ici avec le japonnais, le coréen, un dialecte du centre de la Chine, le thaï, le russe et le télougou). L'utilisation du trill dans certaines langues n'est pas dévoué de sens, on retrouve une certaine motivation à garder l'allophone [r] comme potentielle réalisation de la rhotique que ce soit à travers les chansons, ou encore dans les onomatopées. Il s'agit d'un segment \textg{simil-\textit{r}} que nous ne serions pas surpris de voir apparaître dans des contextes marquées et dans des langues où il ne devrait pas être. \\

Les pressions pour la conservation du trill sont différentes entre les langues. Différentes forces entrent en jeu, créant de la diversité au niveau des rhotiques observées dans le monde, et des différences de distributions pour les trills.
Nous pensons que de futures recherches devraient inclure davantage de langues austronésiennes où la rhotique semble (nous n'avons pas connaissance d'études de corpus détaillées à ce sujet) être très souvent réalisée comme un trill à (au moins) deux contacts.
De même que l'espagnol a pu avoir une influence sur les langues parlées en Amérique du sud, l'indonésien et le malais ont pu avoir ce type d'influence sur les langues parlées en Asie du sud-est.
Mais avant de récolter plus de données, la réflexion entamée dans cette thèse sur la représentation à plusieurs niveaux du trill doit être poursuivie.
De futures recherches devraient aussi être menées sur la diachronie du trill pour en comprendre les origines et les motivations qui ont fait que ce son, bien que complexe articulatoirement, s'est phonologisé dans de nombreuses langues.\\

Finalement, la question initiale de ce doctorat devrait aussi faire l'objet de recherches. Pourquoi certaines personnes ne sont pas capables de faire vibrer la pointe de leur langue pour produire des trills ? Pour ce faire, il faudrait, de même que cette thèse, inclure différents niveaux d'analyse. Ces niveaux pourraient relever de l'acoustique, de l'articulation ou encore du sociétal, dans différentes cultures, des personnes qui n'arrivent pas à produire ce son.

%Différentes pressions entre les langues, différentes forces, variation entre les langues, d'où la typlogie du trill.
%Ouverture sur la diachronie.