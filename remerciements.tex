\section*{Remerciements}

\begin{flushright}
Paʁa Luz Daʁy.\\
%No pude pʁonunciaʁ la eʁe, \\peʁo si pude haceʁ mi tesis sobʁe la eʁe. 
El aʁte de haceʁ una tesis \\sobʁe algo que no puedo pronunciaʁ.
\end{flushright}
~\\


J'ai longtemps réfléchi (eh oui, ça m'arrive de temps en temps de réfléchir), avant d'écrire ces remerciements, à la forme qu'ils devraient prendre. Après tout ce ne sont pas des remerciements de mémoire mais bien de thèse.
Je me suis demandé si je voulais en faire des originaux, des classiques, ou des qui vieillissent bien avec le temps. Ou encore si je voulais les faire chronologiquement, géographiquement, ou encore par ordre d'importance ou encore même par type de personnalité. Je n'ai toujours pas pris de décision, donc je vais les faire en fonction de mon inspiration du moment.\\

Mon histoire débute le jour de l'incendie de Notre-Dame de Paris. C'est le jour où j'ai envoyé un mail à Dan Dediu pour me présenter. Depuis, je suis devenu son doctorant. Merci à toi Dan, de m'avoir pris dans ton projet, de m'avoir accompagné pendant ces trois années (et demi) et de m'avoir partagé ta vision de la science. Merci pour la confiance que tu m'as accordée.

François Pellegrino est arrivé après dans ma vie. Je te remercie pour ta disponibilité et ta sagesse. Merci pour les nombreuses fois où tu es venu me voir pour que j'avance et ne baisse pas les bras, et aussi pour tous les pots cassés que tu as réparés. Sans ta supervision et celle de Dan, ma thèse ne serait pas celle qu'elle est. Merci à vous deux, merci pour votre patience.\\

Je tiens à remercier Ioana Chitoran qui en plus d'avoir accepté d'être la présidente de mon jury de soutenance, a été la personne qui m'a conseillé de contacter Dan il y a 3 ans et demi. Je remercie aussi Laurence Labrune, avec qui j'ai eu la chance d'échanger en colloque, d'avoir accepté d'être rapporteure, et Cédric Gendrot, dont François m'avait fait découvrir le travail, d'être rapporteur.\\

Et c'est à partir de là que ça devient plus difficile de remercier toutes les personnes.

Merci à Marc Allassonnière-Tang, la première personne du laboratoire que j'ai connue, merci pour ta bienveillance, ta bonne humeur, et tes conseils.
Natalja Ulrich, thanks for you support, we managed to start and finish our thesis together, thanks for all these skypes we have done during \textsc{covid} and the trips with your wonderful vintage camping-car. 
Mathilde [maʧild] Josserand, merci pour toute cette charcuterie que tu m'auras apportée (même si mon moi végétarien du futur n'apprécie pas) et merci pour toutes ces belles histoires et anecdotes que tu sais si bien me raconter autour des belles chansons de Mara ou d'Aya Nakamura.\\

Rabia et Christian, Christian et Rabia (pas de favoritisme), merci d'avoir été là et d'avoir rendu cette thèse moins douloureuse. Merci pour tous ces repas en salle de pause, tous ces jeux, tous ces cafés, parce que c'est toujours mieux de papoter avec un bon café. Merci aussi pour toute l'aide administrative et informatique que vous m'aurez fournie. \\

Mes remerciements vont aussi au directeur du laboratoire Dynamique du Langage Antoine Guillaume, ainsi qu'à ses différents membres. Merci à Sophie Bouton, Florence Chenu, Sébastien Flavier, Laurence Husni, Sophie Kern, Anetta Kopecka, Jennifer Krzonowski, Brigitte Pakendorf, Maïa Ponsonnet, Françoise Rose, Alice Vittrant. Certain.e.s ont été dans mes différents comités de suivis de thèse, d'autres ont toujours été disponibles pour m'aider ou me conseiller. Merci à Rathna. Et merci à Christophe dos Santos pour nos discussions d'acoustique, de phonético-phonologie mais aussi pour tes blagues, tes horoscopes et tes recommandations cinématographiques.\\

Je remercie les différentes personnes qui auront côtoyé le meilleur bureau « 214 » avec moi. Ludivine m'aura donné des bons conseils pour bien terminer une thèse et m'aura fait me mettre au jardinage de bureau. Obrigado a Gabriella por sua companhia no escritório e seus biscoitos que comi quando estava com fome. Merci aux stagiaires qui ont partagé ce bureau et qui étaient prêtes à papoter et prendre un petit café en plus de m'aider à travailler sur le /ʁ/ en français. Et à Nelly, petite dernière à arriver dans ce bureau, merci d'être toujours dans le partage. Merci à toi bureau « 214 », ta couleur jaune me manquera. (PS : Mathilde, t'es déjà dans la section un chouïa plus haut mais merci de m'avoir laissé m'étaler sur ton bureau.)\\

Je lève ma tasse de café aux autres doctorantes et doctorants (dont certains sont déjà docteur.e.s) que je suis bien venu déranger quand je n'avais pas envie de bosser. Merci à Jinke pour ces moments passés le soir à nous plaindre et pour m'avoir fait découvrir Wing [la meilleure chanteuse]; Karlito pour tous ces mots doux que tu m'as lus; Krishna for being in the lab during \textsc{covid} time, thank you for your support; Léa pour les BBB (bières, bouffes, barbecues), la luge, notre beau bonhomme de neige, nos discussions sur la vie, et les tarauques; Lucie pour être si efficace, en particulier avec l'organisation du café, et pour ton intérêt pour les rhotiques; Nargil for being my nuna and for taking care of your dongsaeng with food, drinks and motivational quotes; Nichuta pour ta bonne humeur et nos parties de saboteur; Tessa (*lève le bas de sa jambe*) pour tous les desserts gastronomiques que tu auras partagés avec nous; and Tzuyi for your kindness and your sweet words. Et merci aux post-docs : Matt(hew) (\#INFPteam even if I'm far from reaching your wisdom) et Jáyd{\fontspec{Charis SIL}\small e᷈}n (I was supposed to record people saying Jayden with the Jayden intonation but I lacked time). Et aux doctorant.e.s et stagiaires qui viennent et qui partent. Mention spéciale à celles et ceux à qui j'ai emprunté la voix pour étudier leurs trills. I will always be « thirsty », and up for a drink with you'all.\\

Merci à Anna Gwenn Marie Ridez, Mathilde Emma Germaine Josserand, Amatoullah Aouame, Laëtitia Raison-Aubry et Nelly Bonhomme d'avoir relu avec tant d'attention ma thèse. Il faut croire que ma capacité à faire des fautes d’orthographe et de grammaire s'améliore avec le temps. Comme je le disais déjà pour mon mémoire, \textg{[s]’il reste des fautes, c’est entièrement de ma faute car elles ont déjà fait beaucoup pour rendre ce[tte thèse] propre}. Merci à l'incroyable quantité de temps que vous aurez investi dans cette thèse. Thanks to Jayden Macklin-Cordes for proofreading the chapter in English.\\

Ma thèse a eu lieu pendant une période agitée mondialement : comment ne pas mentionner le \textsc{covid}-19, la guerre, et la crise énergétique... Donc je dois aussi remercier mon ami le stress qui ne m'a jamais abandonné dans les moments les plus importants de mon doctorat, et qui a su se faire de plus en plus important dans ma vie.\\

Merci à tous mes amis qui même l'espace d'un repas, d'une aprèm, d'une soirée, d'un week-end, ont toujours été là pour me changer les idées. Merci à Ama, Cha, Dara, Jose, Ludovic, Piña, Phil, Priscillia, Nika, Titish, Thomas et à tous ceux que je ne cite pas explicitement.  Merci à Yiming pour sa compagnie à la BU pendant le \textsc{covid}. Merci aussi à mes colocs qui ont dû me supporter quand je me plaignais en rentrant le soir : Isaac, et Julio.\\

Un grand merci final pour ma famille. Sans toutes les opportunités et le soutien (même de loin), je ne serai pas là aujourd'hui. Merci à vous. Gracias. Petit cœur pour ma grand-mère qui ne comprend toujours pas ce que je fais (PS : moi non plus, je ne suis pas sûr de tout comprendre). ♥\\
Muchas gracias a lxs amigxs de la familia. Gracias a Sergio y Natalia por aceptar ser grabados. Algún día descubriremos por qué algunas personas no son capaces de producir la /r/.\\

\enlargethispage{2\baselineskip}
And finally, thank you {\cjkfont 劉潤燊}, thank you for everything.
\newpage

\hspace{0pt}
\vfill
\begin{flushright}
Cette thèse a été permise grâce à la IDEXLYON Fellowship Grant 16-IDEX-0005 qui m'aura financé pendant trois ans et étendu mon contrat d'un mois et dix jours à cause du \textsc{covid}.
\end{flushright}
\vfill
\hspace{0pt}