\chapter*{Résumé}
\addcontentsline{toc}{chapter}{Resumé}

Cette thèse de doctorat s'intéresse à la représentation du trill alvéolaire pour mieux comprendre sa distribution dans les langues du monde. Le trill est observé sous le prisme de sa représentation dans les transcriptions, dans l'acoustique, dans les grammaires et dans les bases de données phonémiques. Des données issues des \textit{Illustrations of the IPA} publiées dans le \textit{Journal of the International Phonetic Association}, d'enregistrements sonores accompagnant les \textit{Illustrations}, du projet collaboratif Sound Comparisons, de plus de 600 ouvrages décrivant des langues et de PHOIBLE ont été récoltées. Les différentes observations obtenues à partir des données, montrent que, de même que les rhotiques, le trill varie. Les données acoustiques suggèrent que la caractérisation de ce qui est généralement labellisé comme un trill dans les langues du monde n'est pas unique bien que des similarités sur la substance du segment existent entre les langues. Les résultats des différentes analyses sur les langues étudiées confortent l'idée que la réalisation du trill comme un segment à (au moins) deux contacts, n'est pas fréquente contrairement au tap et au flap. Le développement de l'Alphabet Phonétique International, de ses directives, ainsi que la formation des linguistes, a contribué à la surreprésentation du \textit{r}. Les résultats montrent que le trill est, pour la plupart des langues, un allophone peu fréquent, ce qui tend à compliquer la classification des langues en fonction de la présence ou non du segment.