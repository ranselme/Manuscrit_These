\section{Distribution et variation cross-linguistique des sons \textg{simil-\textit{r}}}

\subsection{Distribution dans le monde}

Deux travaux d'importance ont eu comme objectif d'étudier la variation dans les trills avec une approche typologique (avec un recours minimal à des bases de données existantes). Paradoxalement, bien que le thème de ces travaux soit directement relié au contenu de cette thèse, nous n'avons pas eu d'accès direct aux ouvrages à l'exception de \textcite{lindauStory1985}.
Nous allons tout d'abord détailler le manuscript non publié de Jones\footnote{Jones, M. (unpublished). Patterns of variability in apical trills: An acoustic study of data from 19 languages.} puis la thèse de \textcite{inouyeTrillsTapsStops1995} et finalement les travaux de \textcite{lindauStory1985}.\\

Pour cela, nous avons fait le choix de rapporter ces travaux à travers leurs citations dans la littérature. Comme il s'agit de travaux qui étudient plusieurs langues en même temps, nous pouvons supposer que la même méthodologie a été utilisée pour étudier les langues, et donc dans une certaine mesure contrôler la variation entre les productions.\\

\textcite[23]{punnooseAuditoryAcousticStudy2010} mentionne que Jones considère le trill comme central dans les rhotiques, bien que cela puisse amener à l'interprétation fausse qu'une langue avec un /r/ trillé le réalise forcément comme un trill [r]. Jones met en avant de la variabilité dans son étude sur 19 langues due à la complexité articulatoire du segment\footnote{Les différents aspects qui font du trill un segment complexe sont abordés en \autoref{subsec:trill} et dans le \autoref{chap:jipa}}. Les 19 langues utilisées possèdent une seule rhotique. Jones s'intéresse aux rhotiques dans des listes de mots mais aussi dans des histoires \textg{narratives} récoltées en laboratoire. Ses résultats (p. 23 dans \textcite{punnooseAuditoryAcousticStudy2010}) montrent que le trill n'est réalisé comme tel que dans un tiers des cas avec généralement deux occlusions. Les trills sont moins fréquents dans les narratives que dans les listes de mots, et également plus fréquents dans des contextes consonantiques qu'en position intervocalique avec le contexte post-vocalique pré-consonantique favorisé.
Acoustiquement, \textcite[24]{punnooseAuditoryAcousticStudy2010} explique les résultats de Jones par la présence d'un deuxième contact des trills, généralement affaibli. Cela peut se traduire par un manque de détection de la part d'études acoustiques. Enfin, cela donne du crédit à l'hypothèse que ce genre de trill cause une réanalyse perceptuelle menant à un changement du trill au tap. Cependant, pour Jones il existe aussi des réalisations non trillées qui sont des taps. Il ne peut pas affirmer que ces taps soient qualitativement différents des \textg{one-contact trill} (p. 24).\\

Du résumé de la thèse de \textcite{inouyeTrillsTapsStops1995}, nous pouvons dire que l'autrice s'est intéressée à la relation entre trills, taps et stops. L'originalité de sa thèse est d'avoir travaillé sur quelques 65 familles de langues, et d'y avoir inclus des données phonétiques d'environ une quinzaine de langues. Ces données sont des données de première main mais aussi de bases de données déjà existantes (au moins pour les langues australiennes incluses et celles provenant de UCLA phonetic database). L'autrice montre dans sa thèse que les taps comme allophones intervocaliques d'occlusives sont fréquents dans les langues du monde, de même que l'allophone tap en position intervocalique dans les langues décrites avec un trill.\\

\textit{The Story of /r/} de \textcite{lindauStory1985} cherche ce qui caractérise les rhotiques. Pour ce faire, sans détailler sa méthodologie, l'autrice utilise les productions de 92 locuteurs/trices issus de plus de 13 langues/dialectes dont 9 avec un trill [r] comme allophone. Ses données lui permettent de comparer les trills alvéolaires et les trills uvulaires  et d'en dégager les tendances acoustiques. Ses mesures du troisième pic spectal (le troisième formant F3) suggèrent que les productions des trills sont variables chez des locuteurs/trices qui produisent des segments avec un lieu d'articulation plus ou moins alvéolaire, ou plus ou moins dental (de même \textcite{dhananjayaAcousticAnalysisTrill2012} montrent que les trills sont flexibles quant à la position du dos de la langue). La contribution la plus importante de \citeauthor{lindauStory1985} utile à notre thèse est d'affirmer que la réalisation du /r/ comme consonne trillée n'est pas commune même lorsque ces sons ont été décrits comme étant des \textg{trills} (p. 161). En effet, d'autres réalisations sont possibles comme des taps ou des approximantes. \citeauthor{lindauStory1985} met ainsi en avant de la variation à la fois entre les individus (variation inter-individuelle) et au sein des productions des mêmes individus (variation intra-individuelle), qui ne sont pas toutes trillées (sauf pour l'espagnol\footnote{Bien que peu d'occurences de trills en espagnol sont incluses dans notre analyse acoustique, nous retrouvons aussi cette tendance dans le \autoref{chap:acoustics}}).\\
	

\subsection{Exemples de la variation en français et en espagnol autour du monde}

Nous montrons que la variation n'est pas seulement inter-langues mais peut aussi se retrouver au sein de langues parlées à plusieurs endroits du globe avec un héritage colonial. C'est le cas du français et de l'espagnol que nous illustrons dans les sous-sections suivantes.

\subsubsection{Les productions du /r/ dans la francophonie}

Le /r/ français n'est pas uniquement articulé comme une fricative ou une approximante alvéolaire. Ainsi, dans certaines régions du monde francophone il est encore possible de trouver des réalisations de la rhotique alvéolaire. En ce qui concerne les productions uvulaires, elles sont, de même que pour les rhotiques alvéolaires, maîtrisées tardivement par les enfants \parencite{dossantosDeveloppementPhonologiqueFrancais2007,metralCaracterisationAcoustiqueRhotique2021}.\\

Le français s'est exporté dans différentes régions du monde, entraînant des changements spécifiques à chaque région où il a été nouvellement parlé. De manière générale, on retrouve une tendance pour la rhotique apicale à laisser place à la rhotique uvulaire. \textcite{thibaultFrenchOutsideEurope2022} nous résume les différentes études mettant en évidence les variations dans les sons du français en fonction des différentes ères coloniales. Nous présentons les aires géographiques avec les différentes réalisations dans l'ordre donné par \citeauthor{thibaultFrenchOutsideEurope2022}. La variation n'est pas détaillée, avec dans la plupart des cas un ou deux allophones majoritaires décrits à travers leur symbole de l'API.\\

Le français de Saint-Laurent utilise un [r] apical qui est en cours de changement en faveur du [ʁ] uvulaire, de même qu'en français de Montréal \parencite{sankoffInstabilityAlternationMontreal2013,morinApicalUvularWhat2013}.
Le français acadien utilise le flap [ɾ]. Similairement, on retrouve le flap [ɾ] dans le français de Louisiane. À Haïti, en Guadeloupe ou en Martinique, on retrouve différents allophones dont les fricatives [ɣ] vélaire et [ʁ] uvulaire ainsi que l'approximante [w].
Dans les îles de l'océan Indien, en Mauritanie le /r/ est soit supprimé, soit produit uvulairement avec peu d'intensité \parencite[263--264]{ledegenFrenchMauritiusSpeaker2016}.
Au Maghreb, la rhotique a été produite comme un trill alvéolaire [r] mais un changement est en cours et elle est principalement produite comme le [ʁ] uvulaire\footnote{\textcite{thibaultFrenchOutsideEurope2022} référant à Morsly (2003, p. 937) ne mentionne que les productions des hommes sans expliciter ce que les femmes produisent.} Ce changement de la production alvéolaire à l'uvulaire est aussi présent au Liban où les locuteurs jeunes et cultivés préfèrent le [ʁ] uvulaire parisien \parencite[25]{thibaultFrenchOutsideEurope2022}. En Afrique subsaharienne, on retrouve le [r]. 
Dans le français du Djibouti, le flap [ɾ] est présent, dans celui de Madagascar, il s'agit d'un trill [r]\footnote{De même que pour le /r/ du Maghreb, \textcite{thibaultFrenchOutsideEurope2022} réfère à Bavoux (1993, pp. 181-183) pour insister sur le fait que la variante est principalement présente chez les hommes.}. 
Finalement, dans le Pacifique, à Tahiti, le /r/ est réalisé comme un flap alvéolaire ou trill [r].

\subsubsection{Les productions du /r/ dans le monde hispanique}

De même que le français, l'espagnol s'est exporté à l'international.
L'espagnol est parlé principalement en Espagne et en Amérique latine. Semblablement au français, l'espagnol de l'Amérique Latine a été en contact et influencé par de nombreuses langues, notamment des langues locales et des langues parlées par les esclaves venus d'Afrique.\\
%~\\
%/rr/\\
%Fricative pronunciation : Page 12, 31, 81, 138, 140, 171, 189-90, 200, 209, 222-4, 248, 265, 272, 279, 308, 319, 322\\
%Velarizeed pronunciation : 140-1, 333-4\\
%~\\
%/r/\\
%assibilation: 12, 22, 25, 81, 171, 189-90, 200, 209, 22-4, 248, 265, 280, 308, 319, 322\\
%neutralization with /l/: 10-12, 23, 25, 98, 126-8, 139, 168, 187, 200, 211-12, 231-2, 239-40, 271, 283, 299, 322-3, 350
%/tr/\\
%~\\

%Lipski 1991c
%Nunez Cedeno 1990

L'espagnol d'Argentine est caractérisé par un phonème /rr/\footnote{Pour certains auteurs, le /rr/ fait référence à un trill et le /r/ a un tap. C'est le cas ici.} réalisé comme un trill alvéolaire dans le littoral du sud incluant Buenos Aires (p. 170). Au nord, l'espagnol est influencé par les langues locales.
À l'est, l'influence du guarani entraîne une réalisation de type \textg{groove fricative} pour le phonème. Mais dans certains lieux près de la frontière avec le Paraguay, ce dernier est réalisé comme un [ž] (p. 171).
Dans le nord-ouest, le quechua influence l'espagnol mais \textcite{lipski1994latin} ne précise pas les réalisations. \citeauthor{lipski1994latin} ne donne pas d'indications pour l'espagnol de l'Uruguay.\\

En Bolivie, dans les \textg{Altiplano highlands}, le /rr/ est réalisé comme une \textg{groove fricative} ou sibilante qui peut être alvéo-dentale ou prépalatale (p. 189). Le trill [ř] décrit par Gordon\footnote{Gordon Alan (1987) Distribucion demografica de los alofonos de /rr/ en Bolivia. Actas del I congreso International sobre el espanol de America.} (1987, cité par \citeauthor{lipski1994latin}) commence cependant à se généraliser. Dans les \textg{Lowland Llanos}, l'assibilation du /rr/ n'est pas présente bien qu'une tendance aille dans ce sens (Gordon (1987) cité par \citeauthor{lipski1994latin}). Au Chili, l'espagnol est influencé par le mapuche (mapudugnun) (cf. \autoref{subsec:mapuche}) et le /rr/ y est produit comme une \textg{groove fricative}. Au Paraguay, on retrouve une réalisation de trill alvéolaire pour le /rr/ \parencite[308]{lipski1994latin}.\\

En fonction des régions de Colombie, on retrouve différentes influences liées à l'esclavagisme et aux minorités indigènes. Dans les \textg{central highlands}, le /rr/ est un trill faible, avec des réalisations parfois \textg{groove fricatives} là où l'espagnol est influencé par le quechua \parencite[209]{lipski1994latin}. Pour les autres parties de la Colombie, \citeauthor{lipski1994latin} ne donne pas d'indications.
Le quechua a aussi influencé l'espagnol de l'Équateur. Le /rr/ est réalisé comme un trill alvéolaire sauf dans la région \textg{Central highlands}, où il s'agit d'une \textg{groove fricative} similaire à [ž], et dans la région de Cañar et Azuay, où il s'agit d'une fricative \parencite[247--9]{lipski1994latin}. Au Pérou, on retrouve au sud une fricative similaire au [ž], et au nord de la région Andine, c'est le trill qui prédomine similairement à celui du sud de l'Équateur \parencite[320]{lipski1994latin}.
Au Venezuela, on retrouve un trill alvéolaire pour le /rr/ qui peut être partiellement dévoisé \parencite[350]{lipski1994latin}\\

L'espagnol du Costa Rica est constitué de plusieurs dialectes. Dans la \textg{Central Valley}, le /rr/ est caractérisé par une \textg{groove fricative} [ž] qui peut devenir une rétroflexe en locution rapide \parencite[222]{lipski1994latin}. Dans les autres dialectes, on retrouve des fricatives ou sibilantes \parencite[224]{lipski1994latin}. \citeauthor{lipski1994latin} ne donne pas d'indications pour l'espagnol du Salvador et celui du Panama. L'espagnol du Guatemala possède un /rr/ réalisé comme une fricative qui varie d'une fricative prépalatale [ž] à une fricative rétroflexe. Pour le Honduras, les locuteurs éduqués de Tegucigalpa produisent une \textg{groove fricative} pour le /rr/. Au Mexique, on retrouve des influences des langues maya. Le /rr/ y est réalisé comme un trill alvéolaire. La \textg{groove fricative} se retrouve dans le parler des femmes de classe moyenne où la réalisation est vue comme prestigieuse (p. 279). A Chiapas, on retrouve une sibilante comme /rr/. Au Nicaragua, sur la côte, on peut retrouver le /rr/ produit comme un flap ou comme une approximante rétroflexe \parencite[291]{lipski1994latin}. 
À Cuba, l'espagnol a été influencé par les langues parlées par les esclaves d'Afrique. On y retrouve un trill dévoisé pour le /rr/ et une vélarisation possible pour les strates sociales basses \parencite[231]{lipski1994latin}. De même, le trill dévoisé est présent dans l'espagnol de République Dominicaine \parencite[239]{lipski1994latin}. À Puerto Rico, des /rr/ 'vélarisés' sont présents. Il s'agit de productions allant du [x] ou [ʀ] \parencite[333]{lipski1994latin} avec une influence possible soit du français, soit de langues d'Afrique de l'ouest.\\

Cette partie a servi à illustrer la complexité de la réalisation du /r/ et du contact de langue. Ce qui est considéré comme un trill n'est pas forcément trillé car beaucoup de facteurs peuvent intervenir. De même que les langues coloniales ont été influencées par les langues natives, les langues natives peuvent aussi être influencées par les langues coloniales (cf. \autoref{subsec:mapuche}). Le transfert du trill n'est pas que vertical, c'est-à-dire depuis une langue mère vers une langue fille, mais peut aussi être horizontal, c'est-à-dire entre différentes langues. Il s'agit d'un point important à mentionner mais pas crucial pour cette thèse.

