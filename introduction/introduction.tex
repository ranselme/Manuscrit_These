\chapter*{Introduction}
\addcontentsline{toc}{chapter}{Introduction}

%\epigraph{Your're capable of anything because you're so bad at everything.}{Alpha Waymond Wang (Evrything Everywhere All at Once)}

\setlength{\epigraphwidth}{0.6\textwidth}
\epigraph{I don't know. The only thing I do know... is that we have to be kind. Please. Be kind... especially when we don't know what's going on. }{\textsc{Waymond Wang}\\\textit{Everything Everywhere All at Once}}
\vspace{1.5cm}


Roulez-vous vos \textit{r} ? Ce n'est pas mon cas, ni celui de mon directeur de thèse Dan Dediu. Pourtant, une de mes langues natives est l'espagnol, et la langue native de mon directeur est le roumain. Ces deux langues sont généralement décrites comme possédant un \textit{r} roulé. Nous ne roulons pas nos \textit{r} parce que nous ne pouvons pas. Cette question en apparence simple a captivé mon attention pendant les trois dernières années. Qu'est-ce que \textg{rouler} ? Qu'est-ce que \textg{r} ? Et qu'est-ce qu'on fait quand on ne roule pas ses \textg{r} ? Est-ce qu'on peut quantifier le pourcentage d'une population qui ne roule pas ses \textg{r} ? Pour comprendre pourquoi certaines personnes roulent leurs \textg{r} et pourquoi d'autres n'en sont pas capables, nous devons tout d'abord nous intéresser à ce que ce son représente, comment il est représenté, et où on le trouve autour du globe. Cela permettra de mieux envisager des études crosslinguistiques\footnote{Nous utiliserons ce néologisme comme synonyme d'études comparatives entre langues.} sur le \textit{r}. Nous n'utiliserons plus dans le reste de cette thèse l'expression de \textg{\textit{r} roulé} mais celui de \textg{trill}.\\
~\\

Cette thèse s'intéresse au trill ainsi qu'à sa représentation. En un siècle, des données ont été collectées et des modèles linguistiques ont émergés. De plus en plus de langues sont décrites autour du monde, ce qui permet de voir les similarités et les différences entre les langues. Un son qui revient fréquemment dans les descriptions de langues est le trill alvéolaire. Les informations incluses dans les descriptions des langues sont ensuite utilisées en typologie pour expliquer la diversité linguistique. Cela implique d'avoir des concepts à comparer, nous considérerons le trill dans cette thèse comme un concept comparatif \parencite{haspelmathComparativeConceptsDescriptive2010,haspelmathChallengeMakingLanguage2016}.\\


Le trill, représenté à l'aide du symbole \textg{r} dans l'alphabet phonétique international (API) \parencite{barryAnotherRtickle1997,whitleyRhoticRepresentationProblems2003,wieseRepresentationRhoticsRepresentation2011}, est un son produit lorsque la pointe de la langue entre en vibration contre la crête alvéolaire. On parle de trill alvéolaire. Généralement composé de plusieurs périodes, dues à plusieurs contacts de la pointe de la langue contre le palais, ce segment a été décrit comme complexe du point de vue acoustique et articulatoire. Les études en acquisition montrent qu'il est acquis tardivement par rapport à d'autres sons \parencite{mcleodChildrenConsonantAcquisition2018}. Pour autant, sa recherche dans des bases de données montre qu'il est relativement fréquent dans les langues qui possèdent une rhotique \parencite{maddiesonPatternsSounds1984}. Le trill est un son qui appartient aux rhotiques, une classe phonologique qu'on représente avec la lettre <r> et caractérisée par son manque d'unité phonétique \parencite{lindauStory1985,scobbieVariable2006,magnusonStoryTwoVocal2007,chabotWhatWrongBeing2019}.
Le trill, comme nous l'abordons dans cette thèse, est un segment, il appartient donc à la phonétique et phonologie segmentale. Cette thèse s'ancre dans la \textg{taxonomic phonetics} \parencite{ohalaRelationPhoneticsPhonology2010} qui se base sur l'usage de symboles qui ne changent que peu avec le temps. Cet enracinement des symboles a tendance à flouter la frontière entre phonétique et phonologie, alors que les théories en phonétique continuent d'évoluer. L'utilisation de symboles, comme le \textit{r}, reste néanmoins pratique pour chercher des éléments à comparer entre les langues. Nous sommes conscient de la limite de travailler avec un symbole fixe, de sorte que nous abordons aussi le trill à travers sa représentation acoustique.\\


La motivation initiale de cette thèse est de comprendre, dans un premier temps, la distribution des trills à travers les langues du monde et, dans un deuxième temps, pourquoi certaines personnes ne sont pas capables de produire ce son (et être ainsi potentiellement initiatrices de changement linguistique).  Nous chercherons à expliquer pourquoi tant d'auteurs/trices se sont interrogés sur sa possible surreprésentation \parencite{maddiesonPatternsSounds1984,lindauStory1985}.\\


Cette thèse combine plusieurs approches pour appréhender la variation. Nous avons travaillé avec différents supports linguistiques. Nous nous sommes interrogé sur les représentations dans les transcriptions et dans les enregistrements sonores. Du fait de la non-disponibilité des données pour de nombreuses langues, nous avons aussi parcouru des grammaires décrivant des langues parlées par un grand nombre de locuteurs/trices, comme des langues parlées par de petites communautés autour du monde. Dans certains cas, nous avons dû travailler avec des inventaires phonémiques issus de bases de données.\\

Nous n'avons pas inclus de nouvelles données dans cette thèse, autres que celles qui étaient déjà disponibles sur internet. Tous les documents (fichiers d'analyse, tableaux, scripts originaux et modifiés) produits au cours de cette thèse sont disponibles en ligne pour rendre cette recherche un maximum transparente (sur \href{https://github.com/ranselme}{github.com/ranselme)}.
Nous avons eu recours à différentes techniques pour visualiser la variation. Nous avons utilisé différents logiciels comme Praat \parencite{boersmaPRAATSystemDoing2001} pour les analyses acoustiques, Inkscape pour certains graphiques, et RStudio \parencite{rcoreteamLanguageEnvironmentStatistical2020}. Sur R, nous avons pu profiter de nombreux packages comme \texttt{igraph} \parencite{csardiIgraphSoftwarePackage2006} pour les réseaux, \texttt{ggplot2} pour les graphiques \parencite{wickhamGgplot2ElegantGraphics2016a} et \texttt{speakr} pour les représentations du signal acoustique \parencite{corettaSpeakrWrapperPhonetic2022}.

\subsection*{Structure de la thèse}

Cette thèse est composée de la présente introduction, de six chapitres dont l'un est l'objet d'un article accepté dans le \textit{Journal of the Internation Phonetic Association}, et d'une discussion finale.
Après cette introduction générale qui contextualise cette thèse, notre premier chapitre sera dédié à l'état de l'art concernant la description des rhotiques. Nous aborderons la problématique de la représentation et de la caractérisation des rhotiques, et de ce que nous avons appelé sons les \textg{simil-\textit{r}}. Cet état de l'art nous permettra de mettre en avant la terminologie et les représentations changeantes qui rendent l'étude des rhotiques complexe d'un point de vue segmental. La terminologie et les représentations sont importantes. Nous présenterons les différentes rhotiques et nous nous intéresserons aux trills, taps et flaps apicaux, pour lesquels nous décrirons l'articulation et l'acoustique. Nous montrerons que les auteurs convergent à dire que la classe des rhotiques est une classe complexe et pleine de variation à plusieurs niveaux. Nous donnerons des exemples du français et de l'espagnol, où la rhotique, pour différentes raisons, possède différentes réalisations.\\

Nous poursuivrons ce premier chapitre par l'article accepté dans le \textit{Journal of the International Phonetic Association} qui a été coécrit avec Dan Dediu et François Pellegrino. De même que l'article, ce chapitre est écrit en anglais et il peut être lu indépendamment de la thèse. Il se compose d'une introduction, d'un état de l'art, de notre étude sur la base de transcription et d'une discussion. Nous verrons que ce chapitre aborde la problématique de la représentation des rhotiques à travers le choix des symboles qui leur sont dédiés. Nous montrerons que les systèmes de transcriptions ont un rôle à jouer sur notre catégorisation des sons et la surreprésentation du symbole \textit{r}. Nous mettrons en évidence que les transcriptions phonétiques contiennent moins de \textit{r} que les transcriptions phonémiques dans les \textit{Illustrations of the IPA}.\\

Le deuxième chapitre, mettant en valeur des transcriptions de plus de 150 langues, sera complété par le troisième chapitre qui s'appuie sur deux études à partir d'enregistrements sonores issus des \textit{Illustrations of the IPA}. Nous chercherons à observer la variation acoustique dans les trills, taps et flaps. La première étude se base sur une segmentation et une annotation « à la volée », c'est-à-dire à gros grain, alors que la deuxième étude est plus précise sur la segmentation et l'annotation. Cette deuxième étude, portant sur un échantillon de 18 langues, nous permettra de mettre des propriétés temporales. Les résultats obtenus soutiendront l'hypothèse de la surreprésentation des trills.\\

Après ces deux chapitres utilisant les données des \textit{Illustrations of the IPA}, nous utiliserons un nouveau jeu de données issu du projet collaboratif Sound Comparisons. Nous réfléchirons à des méthodes pour comprendre, capturer et visualiser la variation dans les rhotiques des langues d'Europe que nous appliquerons aux branches slaves, romanes et germaniques des langues indo-européennes. Nous mettrons en contraste ces résultats avec les reconstructions de transcriptions phonétiques du latin. De la même manière que avons trouvé de la variation dans les rhotiques pour les langues d'Europe, nous soutiendrons que cette variation a dû exister en latin. Ces différentes études seront accompagnées de données du mapudungun, langue parlée en Amérique du sud, où nous verrons l'influence de l'espagnol sur certaines réalisations de la rhotique.\\

Les deux derniers chapitres s'intéresseront à la place des rhotiques dans les ouvrages de description des langues. Dans le cinquième chapitre, nous répliquerons l'étude crosslinguistique présente dans l'article de \textcite{winterTrilledAssociatedRoughness2022}, où les auteurs montrent que le /r/ trillé est associé avec le concept tactile de \textg{rugosité} à travers les langues du monde. Nous montrerons que l'interprétation de ce qu'est un /r/ trillé n'est pas évidente et proposerons un recodage des données utilisées dans l'analyse d'origine. 
Les divergences observées entre l'étude princeps et notre réanalyse seront interprétées comme une preuve supplémentaire de la difficulté de la catégorisation des rhotiques avec une frontière entre la phonétique et la phonologie floutée dans les grammaires.
Cette réplication sera suivie d'une réflexion sur l'art de décrire, de représenter les sons d'une langue dans les grammaires. Nous montrerons qu'il ne s'agit pas d'une tâche évidente pour les linguistes de terrain, ce qui entraîne de nombreux problèmes pour l'extraction de données issues de ces grammaires. Nous réfléchirons à la place du trill et des autres rhotiques dans ces grammaires pour proposer une visualisation de concepts qu'on retrouve fréquemment dans les grammaires avec les rhotiques. Finalement, nous terminerons ce chapitre en illustrant la complexité de travailler avec des grammaires et des inventaires phonémiques pour établir une typologie du trill.\\

Nous conclurons cette thèse avec le résumé des différents résultats obtenus, ce qui nous permettra de mettre en avant qu'il faut faire attention à la valeur qu'on accorde aux représentations. Nous proposons des perspectives avec de nouvelles études qui s'inscrivent dans la continuité de ce travail.



%\todo[inline]{A finir en fonction de ce qui sera mis dans la conclusion/discussion...}
%La conclusion de cette thèse permettra de... 
%Faire  Surtout pour les études qui n'utilisent pas des sources primaires. Et d'ouvrir les perceptives que nous travail peut apporter à de futur recherches.








